\chapter*{Resumen}

Este Trabajo Final de Grado tiene como misión el diseño y desarrollo de una aplicación cuya principal funcionalidad es, la clasificación de  meteoros mediante sonidos generados por ordenador a partir de detecciones llevadas a cabo mediante radiofrecuencia. 

El proyecto Sonidos del Cielo tiene como objetivo hacer accesible la ciencia a personas con discapacidad visual y para ello, se pretende desarrollar una herramienta mediante la cual se pueda seguir una conversación en \textit{lenguaje natural}, estas herramientas se conocen por el término en inglés \textit{chatbot}. Será necesario que la herramienta sea capaz de reconocer la voz y a su vez transmitir mediante voz la información generada por la aplicación. Esto va a facilitar el acceso a la herramienta no sólo a personas con problemas de visión sino también a niños que aún no sepan leer o escribir.

El diseño completo del proyecto requiere del diseño y desarrollo estructural tanto de una bases de datos en las que se almacenen los elementos necesarios para el proyecto (sonidos de meteoros, curvas de luz, espectrograma, parámetros de detección, etc) como de una API Rest que sirva como controlador y ofrezca servicios entre la base de datos y el chatbot y futuras herramientas que puedan requerirla.

Además, se ha desarrollado una versión preliminar del chatbot para realizar las pruebas necesarias sobre la API Rest y así consolidar las tecnologías y servicios a utilizar en el proyecto Sonidos del Cielo.


%Hasta ahora, se ha realizado el diseño y volcado de la base de datos antigua que contenía la información de las radio detecciones, y se ha comenzado la creación de un API Rest para conectar la base de datos con la aplicación.

%Además, se han mantenido reuniones semanales entre todos los miembros del proyecto, así como la confección de los documentos intermedios y finales del trabajo.


%%--------------
\newpage
%%--------------

\chapter*{Abstract}

The objetive of this Final Degree Project is the design and development an application whose principal funcionality is the classification of meteors using computer-generated sounds from meteor echoes detected by radio frequency.

The Sonidos del Cielo project aims to make science accesible to people with visual disabilities and to achieve this, it is intended to develop a tool through wich you cna have a conversation in natural language, these tools are known as chatbots for the union of the words chat and robot.

It will be necesary for the tool to be able to recognize the voice and at the same time transmit de information generated by the chatbot by voice, wich will facilitate access to the tool not only for people with visual disabilities but also for children who still can't read or write.

The complete design of the project requires the design and structural development of a database in wich all the necessary elements for the project are stores (meteor sounds, light curves, spectrograms, detection parameters, etc) and a REST API that will be used as a controller and will offered services between the database and the chatbot and future tools that may require it.

In addition, a preliminary version of the chatbot has been developed to carry out the necessary test on the REST API and thus consolidate the technologies and services to be used in the Sonidos del Cielo project.


%%%%%%%%%%%%%%%%%%%%%%%%%%%%%%%%%%%%%%%%%%%%%%%%%%%%%%%%%%%
%% Final del resumen. 
%%%%%%%%%%%%%%%%%%%%%%%%%%%%%%%%%%%%%%%%%%%%%%%%%%%%%%%%%%%