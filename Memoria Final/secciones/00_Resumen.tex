\chapter*{Resumen}

Este Trabajo Final de Grado tiene como misión el diseño y desarrollo de una aplicación cuya principal funcionalidad es, la clasificación de  meteoros mediante sonidos generados por ordenador a partir de detecciones de llevadas a cabo mediante radiofrecuencia. 

El proyecto Sonidos del Cielo tiene como objetivo ser accesible a personas con discapacidad visual y para ello, se pretende desarrollar una herramienta mediante la cual se pueda seguir una conversación en \textit{lenguaje natural}, estas herramientas se conocen por el término en inglés \textit{chatbot}.Será necesario que la herramienta sea capaz de reconocer la voz y a su vez transmitir mediante voz la información generada por la aplicación.Esto va a facilitar el acceso a la herramienta no sólo a personas con problemas de visión sino también a niños que aún no sepan leer o escribir.

El diseño completo del proyecto requiere del diseño y desarrollo estructural tanto de una bases de datos en la que se almacenen los elementos necesarios para el proyecto (sonidos de meteoros, curvas de luz, espectrograma, parámetros de detección, etc) como de una API Rest que sirva como controlador y ofrezca servicios entre la base de datos y el chatbot y futuras herramientas que puedan requerirla.

Además se ha desarrollado una versión preliminar del chatbot para realizar las pruebas necesarias sobre la API Rest y así consolidar las tecnologías y servicios a utilizar en el proyecto Sonidos del Cielo.


%Hasta ahora, se ha realizado el diseño y volcado de la base de datos antigua que contenía la información de las radio detecciones, y se ha comenzado la creación de un API Rest para conectar la base de datos con la aplicación.

%Además, se han mantenido reuniones semanales entre todos los miembros del proyecto, así como la confección de los documentos intermedios y finales del trabajo.


%%--------------
\newpage
%%--------------

\chapter*{Abstract}

<<Abstract of the Final Degree Project. Maximum length: 2 pages.>>


%%%%%%%%%%%%%%%%%%%%%%%%%%%%%%%%%%%%%%%%%%%%%%%%%%%%%%%%%%%
%% Final del resumen. 
%%%%%%%%%%%%%%%%%%%%%%%%%%%%%%%%%%%%%%%%%%%%%%%%%%%%%%%%%%%