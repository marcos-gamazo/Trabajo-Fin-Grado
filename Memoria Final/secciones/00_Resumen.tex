\chapter*{Resumen}

Este Trabajo Final de Grado tiene como misión el diseño y desarrollo de una aplicación cuya principal funcionalidad es, la clasificación de  meteoros mediante sonidos generados por ordenador a partir de detecciones de llevadas a cabo mediante radiofrecuencia. 

El proyecto tiene como objetivo ser accesible a personas con discapacidad visual y para ello, se pretende desarrollar una herramienta mediante la cual se pueda seguir una conversación en \textit{lenguaje natural}, estas herramientas se conocen por el término en inglés \textit{chatbot}. Será necesario que la herramienta sea capaz de reconocer la voz y a su vez transmitir mediante voz la información generada por la aplicación.
Esto va a facilitar el acceso a la herramienta no sólo a personas con problemas de visión sino también a niños que aún no sepan leer o escribir.

Por tanto, será necesario realizar un diseño estructural tanto de la base de datos para guardar los distintos parámetros de las radio detecciones, así como de la API que comunique la base de datos con la aplicación y del sitio Web, que deberá tener un diseño \textit{responsive}, es decir, que se adapte a los diferentes dispositivos desde los que se pueda acceder.

Hasta ahora, se ha realizado el diseño y volcado de la base de datos antigua que contenía la información de las radio detecciones, y se ha comenzado la creación de un API Rest para conectar la base de datos con la aplicación.

Además, se han mantenido reuniones semanales entre todos los miembros del proyecto, así como la confección de los documentos intermedios y finales del trabajo.


%%--------------
\newpage
%%--------------

\chapter*{Abstract}

<<Abstract of the Final Degree Project. Maximum length: 2 pages.>>


%%%%%%%%%%%%%%%%%%%%%%%%%%%%%%%%%%%%%%%%%%%%%%%%%%%%%%%%%%%
%% Final del resumen. 
%%%%%%%%%%%%%%%%%%%%%%%%%%%%%%%%%%%%%%%%%%%%%%%%%%%%%%%%%%%