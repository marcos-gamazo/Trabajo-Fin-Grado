\chapter{Conclusiones}

Primeramente, es necesario comentar los resultados finales del Trabajo de Fin de Grado. Desde un primer momento, se tuvo claro que este era un proyecto largo y complicado, puesto que se iba a diseñar e implementar todo el la parte del backend para un proyecto desde cero. Además, no solo había que diseñarlo e implementarlo, sino que se debía hacer cumpliendo una serie de objetivos y requisitos, y además tener cierta capacidad de adaptación para desarrollos futuros.

El objetivo principal del proyecto era la realización de una API REST funcional para su posterior utilización con otros servicios de Sonidos del Cielo. Para cumplir con este requisito también era necesario el diseño y la implementación de una base de datos. En este caso, los líderes de proyecto creyeron conveniente la utilización de una base de datos en MongoDB. La implementación de la API REST ha sido posible gracias al lenguaje de programación Javascript y más en detalle gracias a Node.js y Express, además de la cantidad de módulos existentes que hacen que el desarrollo sobre esta plataforma sea muy cómodo.

Otro de los grandes objetivos, era la creación de un chatbot completamente funcional, que aunque no disponga de capacidad de conversión de texto a voz y de voz a texto, tenga la capacidad de trabajar con la API REST desarrollada con anterioridad. Este requisito también ha sido cumplido y cualquier usuario que utilice la herramienta tiene la capacidad de escuchar el sonido de un meteoro y poder clasificarlo. 

En general, la realización de este Trabajo de Fin de Grado  es muy positivo y además, sirve como base para futuras herramientas entabladas dentro del proyecto Sonidos del Cielo. Lo que quiero recalcar con esto es que los distintos servicios desarrollados hasta ahora, sufrirán mejoras, evoluciones, para que adquieran nuevas capacidades y se puedan cumplir nuevos requisitos y metas establecidas en el proyecto.

La parte negativa del trabajo ha sido la imposibilidad de implementar, por falta de tiempo, un sistema de voz al asistente. Mediante este módulo del chatbot, podríamos comunicarnos con el sin necesidad de utilizar un teclado o un ratón y el asistente poseería la capacidad de contestarnos y guiarnos por el proceso de clasificación.

Centrándome en la parte personal, estoy realmente contento con el resultado del trabajo de fin de grado. Cuando solicité el trabajo lo hice centrado sobretodo en la posibilidad de realizar programación web y también por mi interés con los temas relacionados con el espacio. 
Gracias a esta elección, he conocido el mundo de los chatbots y los variopintos usos de esta tecnología que a día de hoy está disponible en múltiples empresas aunque todavía su uso no está muy extendido. He aprendido como son capaces de clasificar información insertada por el usuario tal y como se explicaba en asignaturas de la carrera. Durante la realización del grado he utilizado en múltiples ocasiones bases de datos relacionales, pero gracias a este proyecto he conocido un poco las curiosidades de las bases de datos no relacionales y el potencial que tienen. 
También he aumentado mis conocimientos sobre programación web gracias a la utilización de Javascript, Node.Js, HTML5 y CSS3. Además de conocer nuevas herramientas para documentar el trabajo como Swagger y utilidades para el despliegue del desarrollo como Docker.

También he reforzado conocimientos adquiridos durante el desarrollo de mi grado en Ingeniería Informática como los sistemas orientados a servicios, la ingeniería del software, las bases de datos, además de poder ver cómo se realiza la planificación de un proyecto, y cómo evoluciona el desarrollo y el trabajo en común de un equipo.

Espero que el desarrollo de este trabajo sirva para acercar la ciencia a todas aquellas personas que habitualmente no tienen trato con ella, personas con discapacidad y especialmente a niños ya que considero de gran importancia despertar en ellos desde jóvenes la pasión y el deseo de trabajar con la ciencia.