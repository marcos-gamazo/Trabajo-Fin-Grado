\chapter{Introducción}
\section{Motivación del proyecto}


La correlación entre hombre y máquina es cada día mayor, avanzando y evolucionando según las necesidades de la sociedad y de los usuarios. En los últimos años se puede observar como se ha ido evolucionando hacia el desarrollo de dispositivos, no necesariamente ordenadores, que estén conectados entre sí, esto se conoce como el Internet de las Cosas (IoT).

Durante la pandemia producida por la COVID-19, hemos podido comprobar las dificultades que existen en el entorno de la educación para poder desarrollar las tareas de una forma normal. Pero ¿y si pudiesemos tener un profesor en la palma de la mano accesible las 24 horas del día los 7 días a la semana?. 

Según una encuesta realizada por el Instituto Nacional de Estadística (INE), entorno al 90\% de los de niños entre 10 y 12 años utilizan ordenadores y navegan por internet de manera habitual \cite{INE}. Hoy en día están muy extendido el uso de asistentes virtuales para la simplificación de tareas. Nuestro objetivo es utilizar estas herramientas para hacer a los niños partícipes  del proyecto y hacerlo de una manera entretenida, mediante técnicas de \textit{gamificación}\footnote{Gamificación: uso de técnicas, elementos y dinámicas propias de los juegos para potenciar la motivación y mejorar el aprendizaje.} nos ayuden a clasificar los distintos tipos de meteoros mediante su sonido.
Esto hace que tengamos que pensar en que no todos los usuarios son iguales, la interfaz de la aplicación no puede ser igual para un niño de 5 años que apenas sabe leer y escribir que para un adulto. Además queremos que sea una aplicación accesible y sirva también para que personas con problemas de visión puedan participar en el proyecto.
Una vez los usuarios hayan clasificado un meteoro un un número determinado de veces; científicos analizarán y comprobarán la validez de las clasificaciones.
  


\vspace*{1.5cm}


%%---------------------------------------------------------